
We have not found similar work on semantic stream classification based on 
clicks. Many traditional document classification methods are based on static 
topic definitions and rely heavily on editorial 
resources~\cite{Sun:2002:WCU:584931.584952}. For instance, web document link 
structures were exploited to predict document categories in the work 
of~\cite{Calado:2006:LSM:1108519.1108533}.
 However, streams change in topical coverage from time to time, and can be 
created or updated on demand to reflect popular need. In our work, document 
classifier is a starting point of our stream classifier. On top of it, the 
stream classifier is rebuilt by optimizing users' click behavior on the 
steam.  





All our models from Phase 0 to Phase 2 are built using targets based on clicks. Using number of clicks as features or target has proved to be effective in many fields, such as recommendation system and search (\cite{Agarwal:2013:CRW:2461256.2461277,Liu:2010:PNR:1719970.1719976}). Some recent work give more up-to-date improvment on using click \cite{Li:2010:CAP:1772690.1772758}.

The logistic regression model used in Phase 0 is a classical supervised approach\cite{ESL}, but the click-based label heuristic appears new. An additional challenge is the number of token features used, which go up in the hundreds of thousands. For the experiments we mainly relied on Vowpal Wabbit \cite{FeatureHashing} and LibLinear \cite{liblinear}, the latter implemented in Weka. 

We also could not find any previous work on the twisted dot-product idea behind Phase 1. However it is closely related to tensor factorization models \cite{rendle2009learning}, \cite{zheng2010flickr}. In our case, the tensor has three dimensions: user, document, and stream, and it is diagonal along each ``stream slice".

Recency is an important metric to consider in recommendation and Web search. 
However promoting recency tends to undercut personalized relevance, making it 
difficult to balance the two. Some pioneering 
work(\cite{Dong:2010:TRR:1718487.1718490}) discussed the problem and proposed 
solutions, where a recency query classifier is used. We deal with recency by 
incorporating the recency feature into a machine learned model.
