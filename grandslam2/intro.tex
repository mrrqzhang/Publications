\section{Introduction}



Developing efficient and accurate personalized recommendation systems of 
news-worthy content has increasingly become a focus of today's media industry. 
On the one hand search engines have enabled users to locate articles or 
multimedia content of any topic within a few keystrokes. On the other hand, 
users are constantly overwhelmed by the amount of information generated 
everyday, much of which is irrelevant to their personal interest. Thus we are 
seeing a resurgence of interest in directory based content discovery systems, 
such as the infinite news streams pioneered at yahoo. 

One of the earliest prototypes of personalized ranking system is based on a 
simple linear matching model of user interest profile and document topic 
profile. This crude approach has proved effective in promoting highly relevant 
articles, videos or slideshows to each individual user, even when we have only 
accumulated partial information about the user. Unfortunately there are 
several drawbacks to this simple model. First the dot product between the user 
profile and document profile is not necessarily a natural one, since both 
profiles are calibrated separately, with no natural complementary ``units" 
relating the two. An immediate remedy is to learn a non-Euclidean inner 
product with suitable objective functions such as click through rate or 
dwell-time. Second, while the linear model presents a ranking order of 
documents based on user's implicit or declared interest, it nonetheless does 
not take into account generic information such as user's age, gender, time of 
the day, day of the week, etc, which could be crucial to users' preference 
among the top documents presented. These are the actual content that the user 
sees, which clearly requires more careful optimization than the entire content 
pool. 

A second challenge goes back to the initial selection of content itself. While 
general news streams certainly occupy the central stage in terms of breadth of 
user coverage, there tends to be less depth and personalization opportunity 
involved in their content. Users who are interested in a specific category of 
news such as sports or finance, or even narrower ones like American football, 
are naturally interested in dedicated streams just for those topics. Indeed 
this is how one accumulates domain-specific expertise as opposed to common 
knowledge. Thus we are interested in constructing appropriate filters 
conforming to prescribed semantic topics. 

In this paper we will address the second problem with a carefully designed 
linear classifier framework based on a combination of user click feedback and 
editorial label. We compare the performance of the classifier approach with 
the baseline TFIDF (token-frequency inverse-document-frequency) approach. 
